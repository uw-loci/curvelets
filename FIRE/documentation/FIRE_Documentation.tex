\documentclass[11pt]{article}
\bibliographystyle{Science}

\usepackage{graphicx}
\usepackage{subfigure}
\usepackage{amsmath}

\textwidth = 6.5 in
\textheight = 9 in
\oddsidemargin = 0.0 in
\evensidemargin = 0.0 in
\topmargin = 0.0 in
\headheight = 0.0 in
\headsep = 0.0 in
\parskip = 0.2in
\parindent = 0.0in


\title{Documentation for the FIbeR Extraction (FIRE) software package}
\author{Andrew M. Stein}

\begin{document}
\maketitle
\date{}

\section{Introduction}
FIRE is a Matlab tool for automatically extracting the network geometry for a stack of confocal microscope images of a semiflexible polymer gel, such as actin, fibrin or collagen.  It works best when the polymer is fluorescently labeled, though less good results can be achieved from reflectance images. The is discussed in detail in:

A.�M. Stein, D.�A. Vader, L.�M. Jawerth, D.�A. Weitz, and L.�M. Sander. An algorithm for extracting the network geometry of 3d collagen gels. \emph{J. Microscopy}, 232(3):463, 2008.

The manuscript is included with this documentation.  The appendix lists the parameters used by the algorithm.  FIRE requires Matlab and the Matlab image processing toolbox.  To test the algorithm, open Matlab and  change to the FIRE/example directory in Matlab. Then, type ``go\_example\_small'' at the Matlab prompt for a quick test (36 s on a 2 GHz MacBook Pro).  For a larger data set, run ``go\_example" (19 min).   When applying FIRE to new data sets, the user is encouraged to start with small images and look at the plots generated by ``go\_example\_small'' to understand what parameters should be altered to improve the performance..

\section*{Acknowledgements}
Thanks to D. Vader and D. Weitz for providing the images on which this has been tested, and L. Sander, L. Jawerth, T. Jackson, P. Smereka, S. Esedoglu, and A. Gilbert for many useful discussions.  This work was supported by NIH Bioengineering Research Partnership grant R01 CA085139-01A2 and the IMA.

\end{document}
